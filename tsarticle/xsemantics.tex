\section{Xsemantics}
\label{sec:xsemantics}

Xsemantics~\cite{lbts} (the successor of Xtypes~\cite{Bet11}) is a DSL (written
in Xtext) for writing type systems, reduction rules, interpreters (and in
general relation rules) for languages implemented in Xtext.
A system definition in Xsemantics is a set of judgment rules which have a
conclusion and a set of premises; these rules can act on any Java object,
though, typically, they will act on EObjects which are elements of the metamodel
of the language implemented in Xtext.  Indeed, Xsemantics relies on Xbase to
provide a rich syntax for defining rules (and premises of rule), thus giving
full access to Java types.

Xsemantics is thought to be used by people who are at least a little familiar
with formal type systems and operational semantics: it aims at providing
a syntax which is close to the way deduction rules are written in a formal
setting~\cite{hindley:1997a,Pierce02}.
Actually, Xsemantics rules are written in the other direction with respect
to standard deduction rules: the conclusion come before the premises; this is just
to make IDE tooling work better, and to give a more "programming" style to rules.

Starting from the definitions of these rules, Xsemantics generates Java code
that can be used in your language implemented in Xtext for scoping and
validation (it also generates a validator in Java).  Xsemantics aims at
providing a rich syntax for defining any kind of rules: relations among elements
(e.g., \emph{subtyping}), \emph{static semantics} (i.e., type systems) and
\emph{dynamic semantics} (i.e., reduction rules that can be used for
interpreting a program).  In this paper, we will use it for writing the rules of
the type system of the case study we are considering
(Section~\ref{sec:casestudy}).

\subsection{Type System Specification}

The first thing to do in a system defined in Xsemantics, after giving it a name,
is to declare the \emph{judgments} of your system; a judgment consists of

\begin{itemize}
\item 
a name, which has to be unique in the system;
\item 
a \textit{judgment symbol} that can be chosen from some predefined symbols;
\item 
the \textit{parameters} of the judgment; parameters of a judgments are separated by
	a \textit{relation symbol} that can be chosen from some predefined symbols;
\end{itemize}

\noindent
The parameters can be

\begin{itemize}
\item 
input parameters, in that case they are declared as Java parameters;
\item 
output parameters, in that case you use the keyword
\verb|output| followed by the Java
type of the output parameter.
\end{itemize}

\begin{lstlisting}[language=xsemantics,float,label=lst:xsemantics-judgments,caption=Judgment
definitions in Xsemantics]
system org.typesys.xsem.guidsl.xsemantics.TypeSystem

import org.typesys.xsem.guidsl.xsemGuiDsl.*

judgments {
	attrtype ||- Attribute attribute : output Type
	exprtype |- Expression expression : output Type
	// whether {@code right} is assignable to {@code left}
	isAssignable |- Type left <~ Type right
	// computes the most general type between {@code first} and {@code second}
	mostGeneral |- Type first ~~  Type second |> output Type
}
\end{lstlisting}

\noindent
The judgment definitions for our case study are shown in
Listing~\ref{lst:xsemantics-judgments}.

Once the judgments of the system are declared, we can start declaring the
rules.  Each rule consists of

\begin{itemize}
\item
a name, which has to be unique in the system;
\item
a \textit{rule conclusion};
\item
the \textit{premises} of the rule;
\end{itemize}

\noindent
The rule conclusion consists of

\begin{itemize}
\item
the name of the \textit{environment} of a rule;
\item
a \textit{judgment symbol};
\item
the \textit{parameters} of the rules, which are separated by
a \textit{relation symbol} that can be chosen from some predefined symbols;
\end{itemize}

The things that make a rule belong to a specific judgment are the judgment
symbol, the relation symbols (which separate the parameters); moreover the types
of the parameters of a rule must be (Java) subtypes of the corresponding types
of the judgment (or exactly the same Java types).  Two rules belonging to the
same judgment must differ for at least one input parameter's type.

The premises of a rule which are specified in a \verb|from| block can be any
Xbase expression, or a rule invocation.  If you think of a rule declaration as a
function declaration, then a rule invocation corresponds to function invocation,
thus you must specify the environment to pass to the rule, and the arguments,
both input and output arguments.
Moreover, if a rule does not require any premise, we can use a special form of
rules, called indeed \textit{axiom}s, which only have a conclusion, without
premises.

At runtime, the system will select the most appropriate rule according
to the runtime types of the passed argument (similar to
\textit{polymorphic dispatch} mechanism).

In the premises you can assign values to the output parameters; and
when you invoke another rule, upon return, the output arguments will have
the values assigned in the invoked rule.

If one of the premises fails, then the whole rule will fail, and in turn all the
stack of rule invocation will fail.  If the premise is a boolean expression, it
will fail if the expression evaluates to false.  If the premise is a rule
invocation, it will fail if the invoked rule fails.

\begin{lstlisting}[language=xsemantics]
axiom SimpleAttributeType
	G ||- SimpleAttribute attr : attr.type

rule DerivedAttributeType
	G ||- DerivedAttribute attr : Type attrType
from {
	G |- attr.expr : attrType
}
\end{lstlisting}

\subsection{Usage in Xtext}

With the declarative specification (written in the Xsemantics DSL), Java methods are generated that can be used in the SopeProvider or Validators. 

If the declarative rules are marked with […], Xtext validator classes are directly created and no further work is necessary.

Example output for reusable methods

\begin{lstlisting}
TypeSystemResult<MyOtherGrammarElement> typeAsMyOtherGrammarElement(MyGrammarElement left); 
TypeSystemResult<MyGrammarElement, MyOtherGrammarElement> type(MyGrammarElement left, MyOtherGrammarElement right); 
TypeSystemResult<Boolean> checkType(MyGrammarElement left, MyOtherGrammarElement right); 
\end{lstlisting}
