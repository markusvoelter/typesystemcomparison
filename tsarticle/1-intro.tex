\section{Introduction}

Today's tools for domain specific languages (DSL) make it feasible to build
complex languages. These usually include data structures, assignments and
expressions which may contain, for instance, boolean comparison operators,
arithmetic operators and function calls. A type system allows to assign
\emph{types} to language elements and specify rules regarding which types are
allowed where and under which conditions. Checking these rules may be
non-trivial, and type systems may help to avoid boilerplate code for validation
and \emph{scoping}. The latter determines visibilities of references and
variables.

Type checking can occur at compile-time which is called static type checking, or
at run-time, for instance when types of variables may only be computed at
run-time. This paper focuses on static type checking, although the type systems
presented here may be used beyond that, e.g. for reduction rules and
interpreters.

A developer of a (domain specific) language may arbitrarily define which of his
model elements he considers to be typeable, what types exist and define
arbitrary rules using these types for what he considers to be a system free of
type errors.
To illustrate the most common tasks of a type system, we use a language for
defining explicitly typed and derived variables, with an expression specifying
how to derive them. It is shown in listing \ref{lst:tlex}. A more detailed
example can be found in \cite{mvdsl}. The example defines a string variable
called \emph{name}, an integer constant and a derived variable called
\emph{greeting} of type string (which happens to be explicitly specified).

\begin{lstlisting}[float,label=lst:tlex,caption=Example for a typed language.]
name : string;
num = 41 + 1;
greeting : string = "Hi " + name + ", the answer is " + num + ".";
\end{lstlisting}

We define the type of the element \emph{string} to be itself, which is usual for
type declarations. We then define the type of variables to be a) the type of the
element after the colon (\verb|:|), if there is one, or, if no colon is there,
the type of the expression after the equals symbol (\verb|=|).
If there are both an explicit type specification and a derivation expression,
the type of the derivation expression has to be conformant to the specified
type. Conformance could mean that the two types have to be in a certain
relationship in the type hierarchy or that the given type is convertible to the
expected type. We define the type of an addition to be the most general type of
its summands and that a string is more general than a number, thus the
derivation expression is of type string and conformant to the explicitly
specified type \emph{string} of \verb|greeting|. In summary, the most common
tasks for type systems are:
Assigning fixed types to language elements, being able to tell whether a type is
conformant to another type, i.e. whether a specific type can be given where
another one is expected, and deriving the types of complex expressions, e.g.
additions.

The rest of this paper is structured as follows. The next section introduces a
common case study which will be implemented in plain Xtext, Xbase, which is
shipped with Xtext 2, Xsemantics \cite{lbts}, and Xtext/TS \cite{mvts}. The
latter two are, like Xtext and Xbase, open source Eclipse plug-ins. Each
implementation will be presented in its own section before comparing them in
the final section.
