\section{Plain Xtext}

To infer types, the plain Xtext scenario implements an operation which determines the actual type of an expression, one for the expected type (which depends on the context the expression is in), and operations which check whether a type is assignable to another type. The Xtext validator in listing \ref{lst:validation-plain} uses these operations. The first check operates on any expression. If the expression is in a context where a certain type is expected, the helper methods are used to check whether the acutal type is assignable to the expected type, for instance \emph{boolean} for the \emph{validate} clause of a widget. 
Otherwise it just returns, e.g. in the case of a derived attribute, whose type is only defined by the derivation expression.

The validations operating on widgets just checks whether the type of the attribute the widget refers to is \emph{boolean}.

\lstinputlisting[label=lst:validation-plain,caption=Xtext validator.,linerange={15-23,31-42,49-57}]{../org.typesys.guidsl/src/org/typesys/guidsl/validation/GuiDslJavaValidator.java}


Xtend 2 

polymorphic dispatch

switch statement

Listing \ref{lst:plain-type-provider} 

The type of a primitive type is the type itself, the type of a simple attribute is the type of its type.

\lstinputlisting[label=lst:plain-type-provider,caption=Type provider in Xtend.,linerange={36-37,46-76}]{../org.typesys.guidsl/src/org/typesys/guidsl/types/GuiDslTypeProvider.xtend}

\lstinputlisting[label=lst:plain-type-conformance,caption=Type conformance specification (Xtend code).,linerange={9-10,15-29}]{../org.typesys.guidsl/src/org/typesys/guidsl/types/TypeConformance.xtend}

