\section{Plain Xtext}

To infer types, the plain Xtext scenario implements operations which determine the actual type of an expression, the expected type (which depends on the context where the expression is used), and operations which check whether a type is assignable to another type. The Xtext validator in listing \ref{lst:validation-plain} uses these operations. The first check operates on any expression. If the expression is in a context where a certain type is expected, helper methods are used to check whether the acutal type is assignable to the expected type, for instance \emph{boolean} for the \emph{validate} clause of a widget. 
In case there is no expected type, for instance for a derived attribute, whose type is only defined by the derivation expression, the check operation just returns.

The validations operating on widgets just check whether the type of the attribute the widget refers to is \emph{boolean}.

\lstinputlisting[label=lst:validation-plain,caption=Xtext validator.,linerange={15-23,31-42,49-57}]{../org.typesys.guidsl/src/org/typesys/guidsl/validation/GuiDslJavaValidator.java}


Xtend 2

polymorphic dispatch

switch statement

Listing \ref{lst:plain-type-provider} shows the recursive computation of types, written in Xtend. The method \emph{getType()} determines the actual type of an expression. The type of a primitive type is defined to be the type itself. The type of an attribute is the type it refers to. In case of a simple attribute it is the type reference and in case of a derived attribute the type of its expression. References to attributes have the type of the referenced attribute. Note that \verb|e.type| is the reference to a (primitive) type while \verb|e.expr.type| and \verb|e.attr.type| use the Xtend 2 syntax shortcut to recursively call \verb|getType(e.expr.type)| and \verb|getType(e.attr.type)|.

The operation \verb|getExpectedType()| calls another operation with the container of the expression under consideration and the 
An example: 

\lstinputlisting[label=lst:plain-type-provider,caption=Type provider in Xtend.,linerange={34-34,36-86,96-98,114-146}]{../org.typesys.guidsl/src/org/typesys/guidsl/types/GuiDslTypeProvider.xtend}

\lstinputlisting[label=lst:plain-type-conformance,caption=Type conformance specification (Xtend code).,linerange={9-10,15-29}]{../org.typesys.guidsl/src/org/typesys/guidsl/types/TypeConformance.xtend}

