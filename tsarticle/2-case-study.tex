\section{Case Study}

In order to compare the different approaches for type systems, we implement the same case study in each variant.

\subsection{The Scenario}
In the case study, entities and forms with widgets for GUIs are modeled. The idea is to model user interfaces for manipulating entities which are, for instance, stored in a database. An entity may have simple attributes of base types like \emph{boolean}, \emph{string}, \emph{int} and \emph{float} and derived attributes with a specification on how to derive them, which may refer to other attributes. The widget of the forms are wired to attributes of the entities and may contain a \emph{validate} clause, which verifies for example the length of the input.

Listing \ref{lst:example-plain} shows an example of the DSL with a \emph{Person} entity and a \emph{PersonForm} to edit it. The attributes \emph{isAdult} and \emph{greeting} are derived. Their type should also be derived.

\lstinputlisting[language=bash,label=lst:example-plain,caption=Forms and Entities DSL,linerange={1-16}]{..//exampleCode/src/plain-xtext.gui}

The Xtext grammar part used by all variants is shown in listing \ref{lst:grammar-plain}. 

\lstinputlisting[label=lst:grammar-plain,caption=Grammar with plain Xtext,linerange={7-53}]{../org.typesys.guidsl/src/org/typesys/guidsl/GuiDsl.xtext}

The type and expressions definition part of the grammar is listed in \ref{lst:grammar-plain-types-and-ex}. It is the same for the Xtext/TS and XSemantics grammars. As Xbase already has support for Java types and expressions, in the Xbase variant, these rules are replaced.

\lstinputlisting[label=lst:grammar-plain-types-and-ex,caption=Grammar with plain Xtext,linerange={54-120}]{../org.typesys.guidsl/src/org/typesys/guidsl/GuiDsl.xtext}

\subsection{Demonstration Tasks}
To compare the different type system variants, each of them will show its type checking ability by demonstrating how it solves the following tasks. 
\begin{enumerate}
\item Ensure that the expression after \emph{validate} is boolean.
\item Show an error message when text widgets refer to boolean entity attributes.
\item Validate that checkbox widgets may only refer to boolean attributes.
\end{enumerate}
All tasks imply being able to infer the type of expressions, as widgets may refer to derived attributes, which are defined with expressions.
