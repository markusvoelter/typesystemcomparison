% This is LLNCS.DEM the demonstration file of
% the LaTeX macro package from Springer-Verlag
% for Lecture Notes in Computer Science,
% version 2.4 for LaTeX2e as of 16. April 2010
%
\documentclass{llncs}
%
\usepackage{makeidx}  % allows for indexgeneration
%

\usepackage{listings} % http://en.wikibooks.org/wiki/LaTeX/Packages/Listings
\usepackage{color}
\definecolor{dkgreen}{rgb}{0,0.6,0}
\definecolor{gray}{rgb}{0.5,0.5,0.5}
\definecolor{mauve}{rgb}{0.58,0,0.82}
\lstset{ %
  language=Java,                % the language of the code
  basicstyle=\footnotesize,           % the size of the fonts that are used for the code
  numbers=left,                   % where to put the line-numbers
  numberstyle=\tiny\color{gray},  % the style that is used for the line-numbers
  stepnumber=2,                   % the step between two line-numbers. If it's 1, each line 
                                  % will be numbered
  numbersep=5pt,                  % how far the line-numbers are from the code
  backgroundcolor=\color{white},      % choose the background color. You must add \usepackage{color}
  showspaces=false,               % show spaces adding particular underscores
  showstringspaces=false,         % underline spaces within strings
  showtabs=false,                 % show tabs within strings adding particular underscores
  frame=single,                   % adds a frame around the code
  rulecolor=\color{black},        % if not set, the frame-color may be changed on line-breaks within not-black text (e.g. commens (green here))
  tabsize=2,                      % sets default tabsize to 2 spaces
  captionpos=b,                   % sets the caption-position to bottom
  breaklines=true,                % sets automatic line breaking
  breakatwhitespace=false,        % sets if automatic breaks should only happen at whitespace
  title=\lstname,                   % show the filename of files included with \lstinputlisting;
                                  % also try caption instead of title
  keywordstyle=\color{blue},          % keyword style
  commentstyle=\color{dkgreen},       % comment style
  stringstyle=\color{mauve},         % string literal style
  escapeinside={\%*}{*)}%,            % if you want to add a comment within your code
  %morekeywords={*,...}               % if you want to add more keywords to the set
}

\usepackage{verbatim}
\newenvironment{code} 
    {\vspace{3mm}
     \noindent
     \rule{\textwidth}{0.7pt}
     \vspace{-4mm}
     \scriptsize \verbatim}
    {\endverbatim \normalsize
	 \vspace{-4mm}
	 \rule{\textwidth}{0.7pt}
	 \vspace{0mm}
	 }

\begin{document}

%
\pagestyle{headings}  % switches on printing of running heads


\title{Type System Frameworks in Xtext 2}

%
\titlerunning{Xtext Type Systems}  % abbreviated title (for running head)
%                                     also used for the TOC unless
%                                     \toctitle is used
%
\author{Markus V\"olter\inst{1} \and Lorenzo Bettini\inst{2} \and Dietmar Stoll\inst{3}}
%
\authorrunning{Terfloth et al.} % abbreviated author list (for running head)
%
%%%% list of authors for the TOC (use if author list has to be modified)
\institute{independent/itemis AG
\and
Dipartimento di Informatica, Universita' di Torino
\and
itemis AG}

\maketitle              % typeset the title of the contribution

\begin{abstract}
\input{abstract.ltx} 
\keywords{Modeling, DSLs, Type Systems, Xtext, Xbase}
\end{abstract}

\section{Introduction}

Today's tools for domain specific languages (DSL) make it feasible to build
complex languages. These usually include data structures, assignments and
expressions which may contain, for instance, boolean comparison operators,
arithmetic operators and function calls. A type system allows to assign
\emph{types} to language elements and specify rules regarding which types are
allowed where and under which conditions. Checking these rules may be
non-trivial, and type systems may help to avoid boilerplate code for validation
and \emph{scoping}. The latter determines visibilities of variables and other
referenceable elements.

Type checking can occur at compile-time (static type checking), or
at run-time, for instance when types of variables may only be computed at
run-time. This paper focuses on static type checking, although the type systems
presented here may be used beyond that, e.g. for reduction rules and
interpreters.

A developer of a (domain specific) language may arbitrarily define which of his
model elements he considers to be typeable, what types exist and define
arbitrary rules using these types for what he considers to be a system free of
type errors.
To illustrate the most common tasks of a type system, we use a language for
defining explicitly typed and derived variables, with an expression specifying
how to derive them. A more detailed example can be found in \cite{mvdsl}. The
example shown in listing \ref{lst:tlex} defines a string variable called
\emph{name}, an integer constant and a derived variable called \emph{greeting}
of type string (which happens to be explicitly specified).

\begin{lstlisting}[float,label=lst:tlex,caption=Example for a typed language.]
name : string;
num = 41 + 1;
greeting : string = "Hi " + name + ", the answer is " + num + ".";
\end{lstlisting}

We define the type of the element \emph{string} to be itself, which is usual for
type declarations. We then define the type of variables to be the type of the
element after the colon (\verb|:|), if there is one. If no colon is there,
it's the type of the expression after the equals symbol (\verb|=|).
If there are both an explicit type specification and a derivation expression,
the type of the derivation expression has to be conformant to the specified
type. Conformance could mean that the two types have to be in a generalization
relationship in the type hierarchy or that the given type is convertible to the
expected type. We define the type of an addition to be the most general type of
its summands and that a string is more general than a number, thus the
derivation expression  of \verb|greeting| is of type string and conformant to
the explicitly specified type \emph{string}. In summary, the most common
tasks for type systems are:
Assigning fixed types to language elements, being able to tell whether a
type is conformant to another type, i.e., whether a specific type can be given
where another one is expected, and deriving the types of complex expressions, e.g.
additions.

The rest of this paper is structured as follows. The next section introduces a
common case study which will be implemented in plain Xtext, Xbase, which is
shipped with Xtext 2, Xsemantics \cite{lbts}, and Xtext/TS \cite{mvts}. All of
these are open source Eclipse plug-ins. Each implementation will be presented in
its own section before comparing them in the final section.
 

\section{Case Study}
\label{sec:casestudy}

As a practial example where type inference and validation is useful, we present a language for modeling forms in a GUI to edit entities in a database. The forms are wired to entities and have widgets like text fields and checkboxes. An entity may have simple attributes of base types like \emph{boolean}, \emph{string}, \emph{int} and \emph{float} and derived attributes. The latter have a specification on how to derive them, which may refer to other attributes. Widgets of forms may contain a \emph{validate} clause, which verifies for example the length of the input. 
In order to compare the different approaches for type systems, the case study is implemented for each of the following variants.

\begin{itemize}
\item Plain Xtext/Xtend
\item Xtext with Xbase
\item Xtext with XSemantics
\item Xtext with Xtext/TS
\end{itemize}

Listing \ref{lst:example-plain} shows an example of the DSL with a \emph{Person} entity and a \emph{PersonForm} to edit it. The attributes \emph{isAdult} and \emph{greeting} are derived, thus it is not necessary to specify their type explicitly. They could be read-only widgets in forms.

%\lstinputlisting[language=bash,label=lst:example-plain,caption=Forms and
% Entities DSL,linerange={1-16}]{..//exampleCode/src/plain-xtext.gui}

%float,
\begin{lstlisting}[language=guidsl,label=lst:example-plain,caption=Forms
and Entities DSL.] 
entity Person {
	name      : string;
	firstName : string;
	age       : int; 
	weight    : float;
	likesCake : bool; 
	isAdult : bool = age > 18;
	greeting = "Hello " + firstName + " " + name + "!";
}

form PersonForm edits Person {
	text (20) -> name validate lengthOf(widgetcontent) >= 2;
	text (20) -> firstName;
	text (5) -> age validate 12.5 > widgetcontent;
	text (5) -> weight validate 0 < widgetcontent;
	checkbox -> isAdult;
	text (30) -> greeting;
}
\end{lstlisting}

%\lstinputlisting[label=lst:grammar-plain,caption=Grammar with plain
% Xtext,linerange={7-53}]{../org.typesys.guidsl/src/org/typesys/guidsl/GuiDsl.xtext}

%float
\begin{lstlisting}[language=xtext,label=lst:grammar-plain,caption=Grammar with plain Xtext.] 
Model:
	(entities+=Entity | forms+=Form )*;

Form:
	"form" name=ID "edits" entity=[Entity] "{"	
		(widgets+=Widget)*
	"}";

Widget:
	TextWidget | CheckBoxWidget ;

TextWidget:
	"text" "(" length=Number ")" "->" attr=[Attribute] 
	("validate" validate=Expression)? ";";

CheckBoxWidget:
	"checkbox" "->" attr=[Attribute] ("validate" validate=Expression)? ";"
;

Entity:
	"entity" name=ID ('extends' superType=[Entity])? "{"
		(attributes+=Attribute)*
	"}";

Attribute:
	name=ID ( ((":" type=Type)? "=" expr=Expression) | 
               (":" type=Type) )";";
\end{lstlisting}

The Xtext grammar part used by all variants is shown in listing \ref{lst:grammar-plain}. The type and expressions definition part of the grammar in listing \ref{lst:grammar-plain-types-and-ex} is the same for the the plain Xtext grammar and the XSemantics and Xtext/TS grammars. The rule order reflects the precedence hierarchy of the operators, from lowest (\emph{BooleanExpression}) to highest (\emph{Atomic}). The rules use \emph{assigned actions} to produce an appropriate abstract syntax tree (cf. Xtext documentation \cite{xtextdoc}), which is used when checking types.
In the Xbase scenario, the rule \emph{Expression} is replaced with the Xbase \emph{XExpression} rule, which maps to Java types and expressions.
%float,
\begin{lstlisting}[language=xtext,label=lst:grammar-plain-types-and-ex,caption=Types and Expressions.] 
Type: PrimitiveType | EntityType;

EntityType: ref=[Entity];

PrimitiveType: NumberType | BooleanType | StringType;

NumberType: FloatType | IntType;

FloatType:   {FloatType}   "float";
IntType:     {IntType}     "int";
BooleanType: {BooleanType} "bool";
StringType:	 {StringType}  "string";

Expression: BooleanExpression;

BooleanExpression returns Expression:
       Comparison
       (({AndOrExpression.left=current} op=("||"|"&&")) right=Comparison)*;

Comparison returns Expression:
       EqualsStuff (({Comparison.left=current} op=(">="|"<="|"<"|">")
       ) right=EqualsStuff)*;

EqualsStuff returns Expression:
       Addition 
       (({Equality.left=current} op=("=="|"!=")) right=Addition)*;

Addition returns Expression:
       Multiplication (({Plus.left=current} '+' | {Minus.left=current} '-') 
        right=Multiplication)*;

Multiplication returns Expression:
       Prefixed (({MultiOrDiv.left=current} op=("*"|"/")) right=Prefixed)*;

Prefixed returns Expression:
       {BooleanNegation} =>"!" expression=Atomic | /* right associativity */
       {ArithmeticSigned} =>"-" expression=Atomic | /* right associativity */
       Atomic;

Atomic returns Expression:
       '(' Expression ')' |
       ({FieldContent} "widgetcontent") |
       ({LengthOf} "lengthOf" "(" expr=Expression ")" ) |
       ({EntityType} "new" ref=[Entity]) | 
       ({BooleanLiteral} value=("true"|"false")) |
       ({FloatLiteral} value=Float) |
       ({IntLiteral} value=INT) |
       ({StringLiteral} value=STRING) |
       ({AttributeRef} attr=[Attribute|ID]);

Float returns ecore::EBigDecimal:   INT? ('.' INT);
Number returns ecore::EBigDecimal: INT ('.' INT)?;
\end{lstlisting}

%\lstinputlisting[label=lst:grammar-plain-types-and-ex,caption=Grammar with
% plain Xtext,linerange={54-120}]{../org.typesys.guidsl/src/org/typesys/guidsl/GuiDsl.xtext}

\subsection{Demonstration Tasks}
To compare the different type system variants, each of them will show its type checking ability by demonstrating how it solves the following tasks.
\begin{enumerate}
\item Ensure that the expression after the \emph{validate} clause is boolean.
\item Show an error message when text widgets refer to boolean entity attributes.
\item Validate that checkbox widgets may only refer to boolean attributes.
\end{enumerate}

All tasks imply being able to infer the type of expressions: The \emph{validate} clause is an expression, and widgets and checkboxes may refer to derived attributes, which are defined with expressions. Test cases for the projects are online at []%TODO


\section{Plain Xtext}

To infer types, the plain Xtext scenario implements operations which determine the actual type of an expression, the expected type (which depends on the context where the expression is used), and operations which check whether a type is assignable to another type. The Xtext validator in listing \ref{lst:validation-plain} uses these operations. The first check operates on any expression. If the expression is in a context where a certain type is expected, helper methods are used to check whether the actual type is assignable to the expected type, for instance \emph{boolean} for the \emph{validate} clause of a widget. 
In case there is no expected type, for instance for a derived attribute, whose type is only defined by the derivation expression, the check operation just returns.

The validations operating on widgets just check whether the type of the attribute the widget refers to is \emph{boolean}.


%class GuiDslTypeProvider {

	@Inject extension TypeConformance conformance
	
	// declare the built-in types for easy use
	Type bool = GuiDslFactory::eINSTANCE.createBooleanType
	Type _float = GuiDslFactory::eINSTANCE.createFloatType
	Type _int = GuiDslFactory::eINSTANCE.createIntType
	Type number = GuiDslFactory::eINSTANCE.createNumberType
	Type string = GuiDslFactory::eINSTANCE.createStringType
	Type primitive = GuiDslFactory::eINSTANCE.createPrimitiveType
	@Inject CyclicDependencyType cyclicType

	def Type getType(EObject e) {
		getType(e, newHashSet())
	}
	
	def Type getType(EObject e, Collection<EObject> visited) {
		if (visited.contains(e)) return cyclicType; // cycle detected
		visited.add(e)
		switch e {
			Widget : e.attr.getType(visited)
			Attribute case e.expr != null && e.type != null 
			   && e.type.isAssignable(e.expr.getType(visited)) : e.type
			Attribute case e.expr != null : e.expr.getType(visited)
			Attribute case e.type != null : e.type
			AttributeRef : e.attr.getType(visited)

	        AndOrExpression : bool 
			Comparison : bool
			Equality : bool
			
			// type is the most general, e.g. int + float => float
			Plus : mostGeneral(e.left.getType(visited), e.right.getType(visited))
			Minus : mostGeneral(e.left.getType(visited), e.right.getType(visited))
			MultiOrDiv case e.op.equals("*"): mostGeneral(e.left.getType(visited),e.right.getType(visited))
			// as in Java
			MultiOrDiv case e.op.equals("/"): e.left.getType(visited)
			
			BooleanNegation : bool
			ArithmeticSigned : number
			
			// return type of attribute referenced by the widget
			FieldContent : return e.getContainerOfType(typeof(Widget))?.attr?.getType(visited)
			LengthOf : _int
			EntityType : e // type is itself
			BooleanLiteral : bool
			FloatLiteral : _float
			IntLiteral: _int
			StringLiteral : string

			default: null
		}
	} 
	def Type getExpectedType(EObject e) { 
		switch e {
			Widget : primitive
			default: internalGetExpectedType(e.eContainer, e.eContainingFeature) 
		}
	} 
	
	def protected Type internalGetExpectedType(EObject e, EStructuralFeature feature) {
		switch e {
			Widget case feature == GuiDslPackage$Literals::WIDGET__VALIDATE: bool
			
			Attribute case e.type != null : e.type
		
			AndOrExpression : bool 
			// an object contained (i.e. left or right side) 
			// in the following operator is expected to always be a number 
			Comparison : number
			// the left side of the operator determines the expected type 
			Equality : e.left.type
			// everything can be added, it might end up as string
			Plus :	mostGeneral(e.left.type, e.right.type).mostSpecific(string)
			Minus      : number
			MultiOrDiv : number

			BooleanNegation : bool
			ArithmeticSigned : number 
			
			LengthOf : string
			
			default : null
		}
	}
%linerange={15-23,31-42,49-57}
\lstinputlisting[label=lst:validation-plain,caption=Xtext validator.,]{code/GuiDslJavaValidator.java}

\subsection{Recursive type computation}

Listing \ref{lst:plain-type-provider1} shows parts of the recursive computation of types, written in Xtend. The method \emph{getType()} determines the actual type of an expression. It avoids endless loops in case of cyclic dependencies by caching already calculated types. The type of a primitive type is defined to be the type itself. The type of an attribute is the type it refers to or the type of its initialization expression. In case there are both an initialization expression and an explicit type, a conformance check is made before. If that check fails, the default (\verb|null|) of the switch statement is returned. References to attributes have the type of the referenced attribute. %Note that \verb|e.type| is the reference to a (primitive) type while \verb|e.expr.type| and \verb|e.attr.type| recursively call \emph{getType()}. They use the Xtend 2 syntax shortcut to \verb|getType(e.expr.type)| and \verb|getType(e.attr.type)|.

\lstinputlisting[label=lst:plain-type-provider1,caption=Type provider in Xtend.]{code/GuiDslTypeProvider.xtend}

The operation \emph{getExpectedType()} in listing \ref{lst:plain-type-provider1} returns the expected type of an expression by checking its context, i.e. the container of the expression. In case of a widget the expected type is always a primitive type, so there's no need to check the context. In other cases, it calls another operation with the container of the expression and the feature of the container pointing to the expression. If the container of an expression is a \emph{Widget} and the reference was the \emph{validate} clause, the expected type is always boolean. The expected type of a subtraction, multiplication and division is a \emph{NumberType}. For an addition, it is a string, unless there is a common type of the summands that is more specific. The method returns \emph{null} to indicate if there is no expected type.

%\lstinputlisting[label=lst:plain-type-provider2,caption=Calculating the expected type.,linerange={96-98,114-146}]{../org.typesys.guidsl/src/org/typesys/guidsl/types/GuiDslTypeProvider.xtend}

\subsection{Type conformance}
An important part of type checking is whether another type can be provided where a certain type is expected.
Listing \ref{lst:plain-type-conformance} shows the Xtend code to compute whether a type is assignable to another, using polymorphic dispatch. The method \emph{isAssignable(left, right)} returns true if one can use an element of type \emph{right} where an element of type \emph{left} is expected. Polymorphic dispatch is used, so that the method with the most specific types is called. The most "general" method has the arguments \emph{(Type, Type)} and specifies that a type can always be assigned to itself, or to one of its supertypes, using the type hierarchy of the EMF model generated by the Xtext Grammar. The \emph{eClass()} method is used to get the EMF \emph{EClass} of a type for comparison (parentheses may be omitted in Xtend 2 if there is no parameter).

An entity may be assigned to an entity if its of the same type (i.e., has the same EMF \emph{EClass}), or if its a super entity. Here, the reference \emph{superType} is not an EMF reference, but the one given in the grammar for \emph{Entity}. Other special cases are dealt with the following methods. As \emph{IntType} and \emph{FloatType} are on the same EMF model inheritance hierarchy level, an assignment rule has to be explicitly specified.

\lstinputlisting[label=lst:plain-type-conformance,caption=Type conformance specification (Xtend code).]{code/TypeConformance.xtend}

\subsection{Summary}
The plain Xtext approach makes use of polymorphic dispatch and the Xtend syntax to keep the code concise, for instance with the Xtend \emph{switch} statement. It consists of four main parts:
\begin{itemize}
\item Xtend code to compute actual types recursively, written in Xtend.
\item An Xtend operation to compute expected types of expressions based on the container they are in.
\item A type conformance specification, written in Xtend.
\item The Xtext validator using the first three.
\end{itemize}



\section{Xbase}
To to leverage type checking and scoping of Xbase in this scenario, the \emph{Expression} rule in the grammar is replaced with the Xbase \emph{XExpression} and \emph{JvmTypes} are inferred for \emph{Entities} and \emph{Forms}.

The validator in the Xbase scenario is shown in listing \ref{lst:xbase-jvm-inferrer}. The validation methods (e.g. the one for a \emph{TextWidget}) look similar to the plain Xtext scenario, with the difference that now \emph{JvmTypes} are compared, which are supplied by a type provider which delegates to the built-in Xbase type provider.

\lstinputlisting[label=lst:xbase-validator,caption=Xtext validator in the Xbase scenario.]	{code/XGuiDslJavaValidatorx.xtend}

\subsection{Java Type Inference}
The specification of how to infer Java types from the DSL is shown in the \emph{JvmModelInferrer} in listing \ref{lst:xbase-jvm-inferrer}.  For each entity, a Java class is inferred with fields, getters and setters. If an expression for an attribute is given, it is interpreted as a derived attribute, so only a getter method is inferred.

Two Xtend syntax shortcuts are used in this class: The extension syntax with the dot after the first argument of an operation (\verb|attribute.toField(...)| instead of \verb|toField(attribute, ...)|, and the closure (\verb|[...]|)after the parenthesis, if it is the last argument of an operation (\lstinline[basicstyle=\ttfamily]{toMethod(...) [...]} instead of \lstinline[basicstyle=\ttfamily]{toMethod(..., [...])}.

To get scoping and type checking of the \emph{validate} clause inside a \emph{Form} from Xbase, a method with return type \emph{boolean} is inferred for each validate clause. The method's name is derived from the attribute's name the widget refers to. The case study requires that the keyword \verb|widgetcontent| in the \emph{validate} clause must refer to the attribute the widget refers to. This is done in Xbase by adding a parameter with the same name of the attribute's type. Though not required in the case study, the entity's attributes are also on the scope by adding a parameter of the entity's type. The parameter name \emph{it} has a special meaning in Xtend - it may be omitted in an expression, so that in a \emph{validate} clause, one could either write \lstinline[basicstyle=\ttfamily]{it.name.length > 2} or \verb|name.length > 2|.

\lstinputlisting[label=lst:xbase-jvm-inferrer,caption=JvmModelInferrer written in Xtend.]	{code/XGuiDslJvmModelInferrer.xtend}

\subsection{Summary}
The Xbase approach uses the \emph{JvmModelInferrer} to infer and generate Java classes with appropriate operations for entities and forms, in order to make use of Xbase for scoping and validation (e.g. for the \emph{validate} clause of a widget). Types are inferred using the Xbase type provider, for example in the validator.


\section{Xsemantics}
\label{sec:xsemantics}

Xsemantics~\cite{lbts} (the successor of Xtypes~\cite{Bet11}) is a DSL (written
in Xtext) for writing type systems, reduction rules, interpreters (and in
general relation rules) for languages implemented in Xtext.
In this paper, we will use it for writing the rules of the type system of the
case study we are considering (Section~\ref{sec:casestudy}).
A system definition in Xsemantics is a set of judgment rules which have a
conclusion and a set of premises; these rules can act on any Java object,
though, typically, they will act on EObjects which are elements of the metamodel
of the language implemented in Xtext.  Indeed, Xsemantics relies on Xbase to
provide a rich syntax for defining rules (and premises of rules), thus giving
full access to Java types.
Starting from the definitions of these rules, Xsemantics generates Java code
that can be used in a language implemented in Xtext for scoping and validation
(it also generates a validator in Java).

Xsemantics is thought to be used by people who are at least a little familiar
with formal type systems and operational semantics: it aims at providing
a syntax which is close to the way deduction rules are written in a formal
setting~\cite{hindley:1997a,Pierce02}.
Actually, Xsemantics rules are written in the other direction with respect
to standard deduction rules: the conclusion comes before the premises; this is
just to make the IDE tooling work better, and to give a more ``programming''-like
style to rules.


% Xsemantics aims at
% providing a rich syntax for defining any kind of rules: relations among elements
% (e.g., \emph{subtyping}), \emph{static semantics} (i.e., type systems) and
% \emph{dynamic semantics} (i.e., reduction rules that can be used for
% interpreting a program).  In this paper, we will use it for writing the rules of
% the type system of the case study we are considering
% (Section~\ref{sec:casestudy}).

A Xsemantics judgment consists of a name, which has to be unique in the system;
a \textit{judgment symbol} that can be chosen from some predefined symbols; the
\textit{parameters} of the judgment; parameters of a judgments are separated by
a \textit{relation symbol} that can be chosen from some predefined symbols.
The parameters can be either input parameters (in that case they are declared as
Java parameters) or output parameters (in that case you use the keyword
\verb|output| followed by the Java type of the output parameter).
% Judgments can have any number of input parameters (though they must have at
% least one input parameter) and at most two output parameters.\footnote{In our
% experience, we never needed more than two output parameters; the extension of
% more than two output parameters is however straightforward.}

\begin{lstlisting}[language=xsemantics,float,label=lst:xsem-judgments,caption=Judgment
definitions in Xsemantics]
import org.typesys.xsem.guidsl.xsemGuiDsl.*

judgments {
	type |- Typable typable : output Type 
	isAssignable |- Type left <~ Type right // whether right is assignable to left
	mostGeneral |- Type first ~~  Type second |> output Type // the most general type between first and second
}
\end{lstlisting}

The judgment definitions for our case study are shown in
Listing~\ref{lst:xsem-judgments}.
Once the judgments of the system are declared, we can start declaring the rules.
 Each rule consists of a name, which has to be unique in the system; a
\textit{rule conclusion}; the \textit{premises} of the rule.  The rule
conclusion consists of the name of the \textit{environment} of a rule; a
\textit{judgment symbol}; the \textit{parameters} of the rules, which are
separated by a \textit{relation symbol} that can be chosen from some predefined
symbols.

The things that make a rule belong to a specific judgment are the judgment
symbol and the relation symbols (which separate the parameters); moreover the
types of the parameters of a rule must be (Java) subtypes of the corresponding types
of the judgment (or exactly the same Java types).  Two rules belonging to the
same judgment must differ for at least one input parameter's type.

The premises of a rule which are specified in a \verb|from| block can be any
Xbase expression, or a \emph{rule invocation}.  If one thinks of a rule
declaration as a function declaration, then a rule invocation corresponds to a
function invocation, thus you must specify the environment to pass to the rule,
and the arguments, both input and output arguments.
The premises of an Xsemantics rule are considered to be in a \emph{logical and}
relation and are verified in the same order they are specified in the block.
If one needs premises (or blocks of premises) in a \emph{logical or} relation,
the operator \verb|or| can be used to separate blocks of premises.
If a rule does not require any premise, we can use a special kind of rule,
called \textit{axiom}, which only has a conclusion.
In the premises you can assign values to the output parameters; and when you
invoke another rule, upon return, the output arguments will have the values
assigned in the invoked rule.

Also the concept of rule environment is taken from the type theory (usually it
is denoted by the $\g$).  It can be used to pass additional arguments to rules
(an empty environment can be passed using the keyword \mykeyb{empty}).
Furthermore, when passing an environment during a rule invocation, one
can specify additional \emph{environment mappings}, using the syntax
\lstinline[breakatwhitespace=false,breaklines=true]!key <- value!, 
where you can use any Xbase expression.
% you can also pass an environment with additional
% mappings separated by a comma (or even build an environment from scratch
% by specifying all the mappings, still separated by a comma); for instance
% \lstinline[breakatwhitespace=false,breaklines=true]!G, x <- 'foo', y <- 10! 
% or
% \lstinline[breakatwhitespace=false,breaklines=true]!x <- o.eClass, y <- (o.eClass.name == 'foo')!, 
% etc.
Note that when you pass an environment to a rule with additional mappings,
you actually pass a brand new environment, thus you will not modify the
current rule environment; if a mapping already exists in the current rule
environment, in the brand new environment (and only there) the existing mapping
will be overwritten.  Thus, the rule environment passed to a rule acts
in a stack manner.



\begin{lstlisting}[language=xsemantics,float,label=lst:xsem-firstrules,caption=Some
examples of rules and axioms in Xsemantics.]
axiom BooleanLiteralType
	G |- BooleanLiteral lit : XsemGuiDslFactory::eINSTANCE.createBooleanType

rule AttributeRefType
	G |- AttributeRef attrRef : Type type
from { G |- attrRef.attr : type }

rule LengthOfType
	G |- LenghtOf len : XsemGuiDslFactory::eINSTANCE.createIntType
from { G |- len.expr : var StringType stringType }

rule FieldContentType
	G |- FieldContent fieldContent : Type type
from { G |- env(G, 'widgetcontent', Attribute) : type }
\end{lstlisting}

In Listing~\ref{lst:xsem-firstrules} we present some rules for the judgment
\mytt{type} (see Listing~\ref{lst:xsem-judgments}, recall that in the rules of
these judgments the second parameter is an output parameter).
For typing a literal (in the example a boolean literal) we write an axiom (since
there is no premise) and the result is a \mytt{BooleanType} (created through the
EMF factory for our language).  The rule for typing an \mytt{AttributeRef} can
be read as follows: the type of an \mytt{AttributeRef} is the type resulting
from typing the corresponding referred attribute (the feature \mytt{attr}, refer
to Listing~\ref{lst:grammar-plain}). The type of a \mytt{LenghtOf} expression is
an integer type, provided that the expression argument of \mytt{LenghtOf} has
string type.  Finally, for typing \mykeyb{widgetcontent} we make use of the
rule environment: we access the environment with the predefined function
\mykeyb{env}, by specifying the key and the expected Java type of the
corresponding value.  If no key is found in the environment or the value cannot
be assigned to the specified Java type the premise will fail.  We will show
how an environment is passed later in Listing~\ref{lst:xsem-validator}.  Thus,
this rule will type \mykeyb{widgetcontent} with the type of the corresponding
attribute. 

In an hypothetical formal type system, we would probably write these typing rules
as

\begin{center}
\begin{scriptsize}
\begin{tabular}{c@{\hspace{.5cm}}c@{\hspace{.5cm}}c@{\hspace{.5cm}}c}
\inferrule
{}
{\g \f \mykeyb{true} : \mykeyb{boolean} }
&
\inferrule
{\g \f \mytt{attr} : \T}
{\g \f \mykeyb{ref} \ \mytt{attr} : \T }
&
\inferrule
{\g \f \mytt{exp} : \mykeyb{string}}
{\g \f \mykeyb{lengthOf}(\mytt{exp}) : \mykeyb{int} }
&
\inferrule
{\g \f \g(\mykeyb{widgetcontent}) : \T}
{\g \f \mykeyb{widgetcontent} : \T }
\end{tabular}
\end{scriptsize}
\end{center}

\noindent
Note how the rules in Xsemantics (see Listing~\ref{lst:xsem-firstrules})
resemble the deduction rules typically used in formal type systems.

At runtime, upon rule invocation, the generated Java system will select the most
appropriate rule according to the runtime types of the passed argument (using
the \textit{polymorphic dispatch} mechanism provided by Xtext).
If one of the premises fails, then the whole rule will fail.
% and in turn all the stack of rule invocation will fail
In particular, if the premise is a boolean expression, it will fail if the
expression evaluates to false.  If the premise is a rule invocation, it will
fail if the invoked rule fails.

% As another example, we show in Listing~\ref{lst:xsem-attribute} the rule for
% \mytt{Attribute}.  Recall (refer to Listing~\ref{lst:grammar-plain}) that an
% attribute can have an explicit type and an initialization expression. The rule
% states that in case the type is not explicit the resulting type is the type of the
% initialization expression; otherwise, the resulting type is the explicit type of
% the attribute; however, in the latter case, if the initialization expression is
% specified, then we must make sure that the type of the expression is assignable
% to the declared type.
% 
% \begin{lstlisting}[language=xsemantics,float,label=lst:xsem-attribute,caption=Type
% rule for \mytt{Attribute}.] 
% rule AttributeType 
% 	G |- Attribute attr : Type attrType 
% from {
% 	if (attr.type != null) {
% 		if (attr.expr != null) {
% 			G |- attr.expr : var Type expType
% 			G |- attr.type <~ expType
% 		}
% 		attrType = attr.type
% 	} else {
% 		G |- attr.expr : attrType
% 	}
% }
% \end{lstlisting}
% 
% Thus, the rule in Listing~\ref{lst:xsem-attribute} relies on rules of the
% judgment \mytt{isAssignable} (see Listing~\ref{lst:xsem-judgments}); 

\begin{lstlisting}[language=xsemantics,float,label=lst:xsem-assignable,caption=Some
rules for the \mytt{isAssignable} judgment.] 
rule IsAssignableBase
	G |- Type left <~ Type right
from { left.eClass == right.eClass }

axiom BooleanAssignableToString    G |- StringType left <~ BooleanType right
axiom IntAssignableToString    G |- StringType left <~ NumberType right
axiom IntAssignableToFloat   G |- FloatType left <~ IntType right

rule EntityTypeAssignable
	G |- EntityType left <~ EntityType right
from {
	left.ref == right.ref
	or
	getAll(right.ref, XsemGuiDslPackage::eINSTANCE.entity_SuperType, XsemGuiDslPackage::eINSTANCE.entity_SuperType, typeof(Entity)).contains(left.ref)
}
\end{lstlisting}

The rules of the judgment \mytt{isAssignable} (see
Listing~\ref{lst:xsem-judgments}), which basically implements subtyping, do not
have an output parameter, they accept two types as parameters; the intention of
these rules is that they succeed if the right parameter is assignable to the
left parameter.
In Listing~\ref{lst:xsem-assignable} we present some rules for the this
judgment.  The first rule is the most general case, and states that types of the
same kind are assignable to each other (in this context ``kind'' corresponds to
EClass); moreover, we have some axioms saying that booleans and integers are
assignable to strings (for instance, like in Java, by an implicit conversion
through \mytt{toString} method).  Finally, an integer can be assigned to a
float.
For the subtyping between two \mytt{EntityType}s the idea is that \mytt{right}
can be assigned to \mytt{left} either if they are the same type (entity
subtyping is reflexive) or if \mytt{left} is a ``super entity'' (possibly
indirectly) for \mytt{right} (entity subtyping is transitive).
To avoid dealing with possible malformed cyclic hierarchies, like in the manual
implementation of the type system (see Listings~\ref{lst:plain-type-provider1}
and~\ref{lst:plain-type-conformance}) we use a predefined function of
Xsemantics, to compute the ``closure'' of a graph:

\begin{lstlisting}[language=xsemantics] 
getAll(eObject, feature to collect, feature to follow, expected type)
\end{lstlisting}

% \noindent An invocation of \mytt{getAll} will return a list of ``expected
% type'', built by collecting all the elements from ``feature to collect'' of the
% specified ``eObject'', and recursively collecting such elements by following the
% feature ``feature to follow'', but avoid possible loops in the EMF graph
% representing the AST.
In Listing~\ref{lst:xsem-assignable}, it will return all
the superclasses of \mytt{right}.  Similarly, we can get all the attributes of
an entity, including the inherited ones, by simply calling

\begin{lstlisting}[language=xsemantics] 
getAll(entity, 
		XsemGuiDslPackage::eINSTANCE.entity_Attribute,
		XsemGuiDslPackage::eINSTANCE.entity_SuperType, typeof(Attribute))
\end{lstlisting}

% \begin{lstlisting}[language=xsemantics,float,label=lst:xsem-mostgeneral,caption=Rule
% for \mytt{mostGeneral}.] 
% rule MostGeneral
% 	G |- Type first ~~ Type second |> Type mostGeneral
% from {
% 	{
% 		G |- first <~ second
% 		mostGeneral = first
% 	} or
% 		mostGeneral = second
% }
% \end{lstlisting}
% 
% The judgment \mytt{mostGeneral} (Listing~\ref{lst:xsem-judgments}) takes two
% types as input parameters and returns as output the most general between the
% two; this judgment has one simple rule, shown in
% Listing~\ref{lst:xsem-mostgeneral}.

\begin{lstlisting}[language=xsemantics,float,label=lst:xsem-binaryexp,caption=Some
rules for binary expressions.] 
rule MinusType
	G |- Minus minus : NumberType type
from {
	// require number types
	G |- minus.left : var NumberType leftType
	G |- minus.right : var NumberType rightType
	// get the most general
	G |- leftType ~~ rightType |> type
}

rule PlusType
	G |- Plus plus : Type type
from {
	// deal with any type
	G |- plus.left : var Type leftType
	G |- plus.right : var Type rightType
	// get the most general (which can also be string)
	G |- leftType ~~ rightType |> type
}
\end{lstlisting}

% \begin{figure}
% \subfloat{\lstinputlisting[language=xsemantics,resetmargins=true]{code/MinusType.xsem}}
% \hspace{20pt}
% \subfloat{\lstinputlisting[language=xsemantics,resetmargins=true]{code/PlusType.xsem}}
% \caption{Global caption}
% \end{figure}

% \begin{listing}[thp]
% \lstset{frame=none}
% \begin{tabular}{ll}
% \begin{lstlisting}[language=xsemantics]
% rule MinusType
% 	G |- Minus minus : NumberType type
% from {
% 	// require number types
% 	G |- minus.left : var NumberType leftType
% 	G |- minus.right : var NumberType rightType
% 	// get the most general
% 	G |- leftType ~~ rightType |> type
% }
% \end{lstlisting}
% \hspace{20pt}
% &
% \begin{lstlisting}[language=xsemantics]
% rule PlusType
% 	G |- Plus plus : Type type
% from {
% 	// deal with any type
% 	G |- plus.left : var Type leftType
% 	G |- plus.right : var Type rightType
% 	// get the most general (which can also be string)
% 	G |- leftType ~~ rightType |> type
% }
% \end{lstlisting}
% \end{tabular}
% \caption{Caption}
% \end{listing}


The rules for \mytt{Minus} and \mytt{Plus} are shown in
Listing~\ref{lst:xsem-binaryexp} (the rules for \mytt{mostGeneral} are not
shown).
The typing rule for \mytt{Minus} requires that the two subexpressions have a
numeric type (recall that since we specify a \mytt{NumberType} as the output
argument in rule invocation, its invocation will succeed only if the result is
assignable to \mytt{NumberType}); the resulting type will be the most general
type, thus, for instance, if one of the two subexpressions has the type
\mytt{FloatType} and the other one \mytt{IntType}, the resulting type will be
\mytt{FloatType}.  Recall that we use \mytt{+} not only as the arithmetic
operator, but also for string concatenation; in particular, if one of the
subexpression is a string, the whole expression is considered as a string
concatenation.  Thus, the rule for \mytt{Plus} computes the types of the two
subexpressions, and then gets the most general; if one of them is a string type,
the whole expression will have string type (see also subtyping rules in
Listing~\ref{lst:xsem-assignable}).

% \subsection{Rules for the Validator}
% \label{sec:xsem-validation}

In a Xsemantics system we can specify some special rules,
\mytt{checkrule}, which do not belong to any judgment. They are used by
Xsemantics to generate a Java validator for the Xtext language.  A
\mytt{checkrule} has a name, a single parameter (which is the EObject which will
be checked by the validator) and the premises (but no rule environment).  The
syntax of the premises of a \mytt{checkrule} is the same of standard rules.
Xsemantics will generate a Java validator with a \checkm{} method for each
\mytt{checkrule}; just like in Java validators for Xtext languages, you can have
many checkrules for the same JavaType (provided the rule name is unique).

\begin{lstlisting}[language=xsemantics,float,label=lst:xsem-validator,caption=Some
checkrules for the Validator.] 
checkrule AttributeTypeChecks for Attribute attribute
from { empty |- attribute : var Type type }

checkrule ValidateMustBeBoolean for Widget widget
from {
	widget.validate == null
	or 
	'widgetcontent' <- widget.attr |- widget.validate : var BooleanType boolType
}

checkrule ValidateTextWidgetAttributeNotBoolean for TextWidget widget
from {
	'widgetcontent' <- widget.attr |- widget.attr : var Type attrType
	!(attrType instanceof BooleanType)
}

checkrule ValidateCheckBoxWidgetAttributeBoolean for CheckBoxWidget widget
from { 'widgetcontent' <- widget.attr |- widget.attr : var BooleanType attrType }
\end{lstlisting}

In Listing~\ref{lst:xsem-validator} we present some checkrules for validating
the elements of our language (see Section~\ref{sec:casestudy}).  The first
checkrule basically states that an Attribute is correct if we can give it a type
(in the empty environment).  The second one accepts a Widget and ensures that
either its validate part is not specified or it has a boolean type; note that in
this case we pass to the type rule invocation an explicit environment so that we
are able to type possible occurrences of \mykeyb{widgetcontent}.  These first
two rules also show an important use of the environment: since the rule for
typing \mykeyb{widgetcontent} (Listing~\ref{lst:xsem-firstrules}) requires that
the string `widgetcontent' is bound to an attribute in the environment, and
since when typing an attribute we provide an empty environment, then a possible
occurrence of \mykeyb{widgetcontent} in an attribute's initialization expression
(which is accepted by the grammar) will be automatically (and correctly)
rejected.
The third checkrule requires that the TextWidget's attribute is not of boolean
type, while the fourth one requires that the checkbox's attribute has a boolean
type (by implicitly trying to assign the result type, in the rule invocation, to
a boolean type).


\section{XTS}
\label{sec:xts}

\subsection{Introduction}

XTS was originally developed as a framework. The type system rules  had been
specified declaratively via a Java API. Later, the DSL was put on top of the
framework. The DSL simply generates the Java code you would have written
manually without the DSL. For some aspects of type systems you still have to
write procedural Java code, because no suitable declarative abstraction is
available, either in the framework or in the DSL.

At the center of XTS is the ITypesystem interface. It has various methods for
calculating the type of model elements, and for comparing types for
compatibility and subtyping relationships. In principle, you can implement this
interface anyway you like. however, to benefit from the framework, you should
use the declarative APIs used in the DefaultTypesystem implementation.

\subsection{Hooking up the type checker}

To enable type checks, you have to hook up the type system framework  with the
validation  framework provided by Xtext. Here is the code you have to write in
the validator:

\begin{lstlisting}[language=Java] 
@Inject private ITypesystem ts;

@Check(CheckType.NORMAL)
public void validateTypes( EObject m ) {
    ts.checkTypesystemConstraints( m, this );
}    
\end{lstlisting}

The type system can be used, mostly in the same way as shown above, in any other
place in an Xtext DSL implementation.
For example it has been used as part of scope implementations. In this case, you
simply inject the type system into the scope provider and use it from there.

\subsection{Setting up}

Note: to be able to use the type system DSL you have to download and install the
type system plug-ins from TODO

We start with the header of the type system file.

\begin{lstlisting}[language=xts] 
typesystem org.typesys.xts.guidsl.typesys.GuiDlsTypesystem 
    ecore file 
    "platform:/resource/org.typesys.xts.guidsl/src-gen/org/typesys/xts/guidsl/GuiDsl.ecore"
    language package org.typesys.xts.guidsl.guiDsl.GuiDslPackage 
\end{lstlisting}

The header of the type system file specifies the fully qualified class name of
the type system implementation file generated from this specification file. We
also have to provide the platform URI for the Ecore file that contains the
metaclasses for which we want to specify the type system rules. Finally, we have
to provide the the fully qualified name  to the package class  generated from
that Ecore file.

\subsection{The type of types}

Type system specifications  are structured into sections. They have no meaning
beyond structuring the overall file. Within this section we can define typeof
clauses. A typeof clause defines how the type for a given metaclass is
calculated, and can optionally specify constraints on the types of properties of
these metaclasses.

In the initial section in the following piece of code we specify  that the type
of the Type metaclass and all its subclasses (hence the +) is a clone of itself.
In other words, types of their own types.

\begin{lstlisting}[language=xts] 
section "Types"
    typeof Type+ -> clone
    subtype IntType base FloatTyoe
\end{lstlisting}

We also specify the subtyping relationship between FloatType and IntType. This
means that, wherever a FloatType is expected, an IntType can also be used.  But
not the other way around. In other words, IntType is more specialized.

\subsection{Literals}

The type of string literals and Boolean literals is always the same, so it can
be fixed to a specific type.
However, for number literals it is more complicated:
whether it is an integer or float type depends on the value and this cannot be
expressed declaratively in the DSL. So we declare the type for NumberLiteral to be calculated with Java
code.

\begin{lstlisting}[language=xts] 
section "Literals"
      typeof StringLiteral -> StringType
      typeof BooleanLiteral -> BooleanType
      typeof NumberLiteral -> javacode
\end{lstlisting}
 
This specification leads to the generation of an abstract method into the
generated type system class, which we have to override in the manually written
subclass. The corresponding method looks as follows:
 
\begin{lstlisting}[language=Java] 
public EObject type( NumberLiteral s, TypeCalculationTrace trace ) {
    if ( s.getValue().equals(s.getValue().intValue())) {
        return create(cl.getIntType());
    }
    return create(cl.getFloatType());
} 
\end{lstlisting}



\subsection{Expressions}

Before we define the type system rules for the various expressions,  we first
define two characteristics. A \emph{characteristic} is essentially a set of
types:
instead of listing the set of types over and over again, we can use the
characteristic the shortcut.

\begin{lstlisting}[language=xts] 
characteristic COMPARABLE {
    IntType, FloatType, BooleanType, StringType
}  
  
characteristic NUMERIC {
    IntType, FloatType
} 
\end{lstlisting}

Then we define the abstract type for the Expression class: such class
has no type, since it is itself abstract, and all it subclasses have different
types. However, it makes sense to declare this fact explicitly, because the type
system DSL editor can then check that all subclasses of Expression are actually
covered by type system rules.

\begin{lstlisting}[language=xts]
typeof Expression -> abstract
\end{lstlisting}

We can now take a look at some of the more interesting cases. For comparisons,
the left and right arguments  have to be comparable (see above). In addition,
they also have to be compatible. For instance, while boolean and string types
are both comparable, they cannot be compared to each other. This is why we need
this explicit compatibility check:

\begin{lstlisting}[language=xts]
 typeof Comparison -> BooleanType {
     ensureType left :<=: char(COMPARABLE)
     ensureType right :<=: char(COMPARABLE)
     ensureCompatibility left :<=>: right
 } 
\end{lstlisting}


The symbol \verb|:<=:| is called ordered compatibility. It
means that the type on the left must be the same or a subtype of the type
specified on the right. The symbol \verb|:<=>:| represents unordered
compatibility: the left must be the same or a subtype of the right, or vice
versa.

\begin{lstlisting}[language=xts,float,label=lst:xts-binaryexpressions,caption=Some
rules for expressions.] 
typeof Equality -> BooleanType {
    ensureType left :<=: char(COMPARABLE), BooleanType
    ensureType right :<=: char(COMPARABLE), BooleanType
    ensureCompatibility left :<=>: right
}

typeof AndOrExpression -> BooleanType {
    ensureType left :<=: BooleanType
    ensureType right :<=: BooleanType 
}   

typeof Plus -> common left right {
    ensureType left :<=: StringType, char(NUMERIC)
    ensureType right :<=: StringType, char(NUMERIC)
    ensureCompatibility left :<=>: right
} 

typeof Minus -> common left right {
    ensureType left :<=: char(NUMERIC)
    ensureType right :<=: char(NUMERIC)
    ensureCompatibility left :<=>: right
} 

typeof MultiOrDiv -> common left right { 
    ensureType left :<=: char(NUMERIC)
    ensureType right :<=: char(NUMERIC)
} 

typeof BooleanNegation -> BooleanType {
    ensureType expression :<=: BooleanType
}
\end{lstlisting}

The remainder of specifications for the binary expressions is essentially more
of the same (Listing~\ref{lst:xts-binaryexpressions}).
The only thing worth mentioning is the \mykeyb{common} keyword. Using common in the
type for a meta class means that the type is going to be the common supertype of the two
arguments. This only works, if the two types are either the same or top of a
subtype relationship (such as float and int).

Let us now see some more special cases: the type of the attribute reference
is the type of the referenced attribute. In other words, we have to follow the 
attr reference to find out the type:

\begin{lstlisting}[language=xts]
typeof AttributeRef -> feature attr
\end{lstlisting}

The type of the widgets (although they are not expressions, but as we will see
below, it is useful if they have a type) should also be relatively
straigthforward.
This is also where  the type system rules for our  test case are  finally
implemented (Listing~\ref{lst:xts-widgets}).

\begin{lstlisting}[language=xts,float,label=lst:xts-widgets,caption=Rules
for widgets.] 
// 1) the expression after "validate" must be boolean
typeof Widget -> abstract

// 2) text widgets may only refer to non-boolean attributes 
typeof TextWidget -> feature attr {
    ensureType length :<=: IntType
    ensureType attr :<=: StringType, IntType, FloatType
    ensureType validate :<=: BooleanType
}  

// 3) checkbox widgets may only refer to boolean attributes
typeof CheckBoxWidget -> feature attr {
    ensureType attr :<=: BooleanType
    ensureType validate :<=: BooleanType
}
\end{lstlisting}


We can now calculate the type of the field content expression, which has to have
the same type as  the attribute to which the owning widget points. Since we
have defined the type of the widget to be the type of the reference attribute,
we can now use the following type specification for FieldContent:

\begin{lstlisting}[language=xts]
typeof FieldContent -> ancestor Widget
\end{lstlisting}

The type of NewExpr is the type of the new'ed entity:

\begin{lstlisting}[language=xts]
    typeof NewExpr -> feature entity
\end{lstlisting}

Finally, the type of an Entity also has to be calculated with Java code, because
it has to be an entity type that references the corresponding entity:

\begin{lstlisting}[language=Java]
protected boolean compareTypes( EntityType t1, EntityType t2, CheckKind k, TypeCalculationTrace trace ) {
    if ( k == CheckKind.same ) return t1.getRef() == t2.getRef();
    if ( k == CheckKind.ordered ) return t1.getRef().getExtends() == t2.getRef();
    return false;
}
\end{lstlisting}


There is one more interesting open issue. We have to implement the subtyping
relationship between entities. This is not so simple, because we will compare 
EntityTypes, and the subtyping depends on whether their corresponding referenced
entities are subtypes of each other. So instead of declaring a subtype relationship
in the DSL, we can implement a type comparison function in Java:

\begin{lstlisting}[language=Java]
protected boolean compareTypes( EntityType t1, EntityType t2, CheckKind k, TypeCalculationTrace trace ) {
    if ( k == CheckKind.same ) return t1.getRef() == t2.getRef();
    if ( k == CheckKind.ordered ) return extends( t1.getRef(), t2.getRef() );
    return false; 
}
\end{lstlisting}








The type of the FieldContent expression also has to be calculated with Java
code, because it involves relatively complicated model navigation that cannot be
expressed declaratively.








\subsection{Learn more}

XTS docs
  

\section{Comparison}

Comparison Criteria

Table

%Lines of Code added (Java, Xtend, DSL?)?

\section{Conclusion}


\bibliographystyle{abbrv} 
\bibliography{lit.bib}
\end{document}
