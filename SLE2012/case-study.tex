\section{Case Study}
\label{sec:casestudy}

As a practical example where type inference and validation is useful, we present
a language for modeling forms in a GUI to edit entities in a database.  The
presented language is not intended to have all the features to be used in
practice: it is only used as a case study to implement type systems, thus we
will concentrate on the features that are more interesting to this aim.

An entity can have attributes of base types like \emph{boolean}, \emph{string},
\emph{int} and \emph{float}, or entity types.
Attributes can have an explicit type and an initialization expression.
If the attribute has an explicit type and an initialization expression, we
require that the (inferred) type of the expression is a subtype of the
attribute's type. If the attribute has no explicit type then the initialization
expression is mandatory and used to infer the type of the attribute.
Since entities can specify an entity to derive from, we also have a subtyping
relation on entity types implied by the transitive closure of the
\mykeyb{extends} relation.

The forms are wired to entities and have widgets like text fields and
checkboxes which refer to a specific attribute of the entity.
Widgets of forms may contain a \emph{validate} clause, which verifies, for
example, the length of the input.
In widgets' validate expressions, one can access the attribute by means of
\mykeyb{widgetcontent}, which will then have the type of the corresponding
attribute.  Listing~\ref{lst:example-plain} shows an example of the DSL with a
\emph{Person} entity and a \emph{PersonForm} to edit it. Note that the attribute
\emph{isAdult} has an explicit type, and its initialization expression
conforms to that type; \emph{greeting} has no explicit type, and its type is
inferred from its initialization expression.
%They could be read-only widgets in forms.

%\lstinputlisting[language=bash,label=lst:example-plain,caption=Forms and
% Entities DSL,linerange={1-16}]{..//exampleCode/src/plain-xtext.gui}

%float,
\begin{lstlisting}[language=guidsl,float,label=lst:example-plain,caption=Forms
and Entities DSL.] 
entity Person {
	name      : string;
	firstName : string;
	age       : int; 
	weight    : float;
	likesCake : bool; 
	isAdult : bool = age > 18;
	greeting = "Hello " + firstName + " " + name + "!";
}

form PersonForm edits Person {
	text (20) -> name validate lengthOf(widgetcontent) >= 2;
	text (20) -> firstName;
	text (5) -> age validate 12.5 > widgetcontent;
	text (5) -> weight validate 0 < widgetcontent;
	checkbox -> isAdult;
	text (30) -> greeting;
}
\end{lstlisting}


% In order to compare the different approaches for type systems, the case study is
% implemented for each of the following variants.
% 
% \begin{itemize}
% \item Plain Xtext/Xtend (Section~\ref{sec:plain-xtext})
% \item Xtext with Xbase (Section~\ref{sec:xbase})
% \item Xtext with XSemantics (Section~\ref{sec:xsemantics})
% \item Xtext with Xtext/TS (Section~\ref{sec:xts})
% \end{itemize}

%\lstinputlisting[label=lst:grammar-plain,caption=Grammar with plain
% Xtext,linerange={7-53}]{../org.typesys.guidsl/src/org/typesys/guidsl/GuiDsl.xtext}

%float
\begin{lstlisting}[language=xtext,float,label=lst:grammar-plain,caption=Grammar
with plain Xtext.] 
Model: (entities+=Entity | forms+=Form )*;
Entity: "entity" name=ID ('extends' superType=[Entity])? "{" (attributes+=Attribute)* "}";
Attribute: name=ID ( ((":" type=Type)? "=" expr=Expression) | (":" type=Type) )";";
Form: "form" name=ID "edits" entity=[Entity] "{" (widgets+=Widget)* "}";
Widget: TextWidget | CheckBoxWidget ;
TextWidget:
	"text" "(" length=Number ")" "->" attr=[Attribute] ("validate" validate=Expression)? ";";
CheckBoxWidget: "checkbox" "->" attr=[Attribute] ("validate" validate=Expression)? ";" ;
\end{lstlisting}

The Xtext grammar part used by all variants of type system implementation is
shown in Listing~\ref{lst:grammar-plain}. The type and expression definition
part of the grammar in Listing~\ref{lst:grammar-plain-types-and-ex} is the same
for the the plain Xtext grammar and the XSemantics and Xtext/TS grammars (for lack of space we
do not show the complete grammar of expressions).
The rule \mytt{Typable} is not called by any other rule of the grammar: it is
only useful to create a common superclass for \mytt{Attribute} and
\mytt{Expression} which are the elements we want to type.
In the Xbase scenario, the rule \emph{Expression} is replaced with the Xbase
\emph{XExpression} rule, which maps to Java types and expressions.

\begin{lstlisting}[language=xtext,float,label=lst:grammar-plain-types-and-ex,caption=Types
and Expressions.] 
Typable: Attribute | Expression;

Type: PrimitiveType | EntityType;
EntityType: ref=[Entity];
PrimitiveType: NumberType | BooleanType | StringType;
NumberType: FloatType | IntType;
FloatType:   {FloatType}   "float";
IntType:     {IntType}     "int";
BooleanType: {BooleanType} "bool";
StringType:	 {StringType}  "string";

Expression: BooleanExpression;
/* skipped full Expression grammar */
Atomic returns Expression:
       '(' Expression ')' |
       ({FieldContent} "widgetcontent") |
       ({LengthOf} "lengthOf" "(" expr=Expression ")" ) |
       ({EntityType} "new" ref=[Entity]) | 
       ({BooleanLiteral} value=("true"|"false")) |
       ({FloatLiteral} value=Float) |
       ({IntLiteral} value=INT) |
       ({StringLiteral} value=STRING) |
       ({AttributeRef} attr=[Attribute|ID]);
\end{lstlisting}

%\lstinputlisting[label=lst:grammar-plain-types-and-ex,caption=Grammar with
% plain Xtext,linerange={54-120}]{../org.typesys.guidsl/src/org/typesys/guidsl/GuiDsl.xtext}


% The rule order reflects
% the precedence hierarchy of the operators, from lowest
% (\emph{BooleanExpression}) to highest (\emph{Atomic}). The rules use
% \emph{assigned actions} to produce an appropriate abstract syntax tree (cf.
% Xtext documentation \cite{xtextdoc}), which is used when checking types.
% float,


To compare the different type system variants, in each of them we implemented
the following checks: the attribute's initialization expression (if present)
is conformant to the attribute declared type (if specified); the expression
after the \emph{validate} clause is boolean; the text widgets do not refer to
boolean entity attributes; the checkbox widgets refer only to boolean
attributes.
All tasks imply being able to infer the type of expressions: The \emph{validate}
clause is an expression, and widgets and checkboxes may refer to attributes with
initialization expressions.  Due to lack of space we cannot show all the code of
the different implementations; we refer the interested reader to the complete
implementations source code:
\url{https://github.com/markusvoelter/typesystemcomparison}.
% Test cases for the projects are online at []%TODO
