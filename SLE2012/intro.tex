\section{Introduction}
\label{sec:introduction}

Developing a compiler for a language and its integration in Eclipse is usually
time consuming since it requires many phases, starting from parsing the program,
checking that is correct, up to the generation.  Furthermore, these mechanisms
have to integrated in the Eclipse IDE, which, in turn requires more manual
programming in order to provide background parsing of the program, error marker
generation, and all the tooling mechanisms to give the programmer a good
experience.  Xtext~\cite{xtext} is a framework for the development of
programming languages as well as other domain-specific languages (DSLs), which eases all
these tasks: it provides high-level mechanisms that generate all the typical and
recurrent artifacts necessary for a fully-fledged IDE on top of Eclipse.

Thus Xtext makes it easier to build DSLs.  However, for complex languages, the
type checking phase still requires additional effort.
A \emph{type system} allows to assign \emph{types} to language elements and
specify rules regarding which types are allowed where and under which
conditions.
Checking these rules may be non-trivial, and type systems may help to avoid
boilerplate code for validation and \emph{scoping}. The latter determines
visibilities of variables and other referenceable elements.
Type checking can occur at compile-time, or at run-time, for instance when types
of variables may only be computed at run-time.   A developer of a language may
arbitrarily define which of his model elements he considers to be typable, what
types exist and define arbitrary rules using these types for what he considers
to be a system free of type errors.
Besides type computation, \emph{type conformance} or \emph{subtyping}, i.e., the
property that two types have to be in a generalization relationship in the type
hierarchy or that the given type is convertible to the expected type, is another
important issue in a type system.
In summary, the most common tasks for type systems are:
assigning fixed types to language elements, being able to tell whether a type is
conformant to another type, i.e., whether a specific type can be given where
another one is expected, and deriving the types of complex expressions, e.g.
additions.

Thus, the contribution of this paper is to show how a type system can be
implemented in Xtext for languages which require non-trivial checks based on
types, in particular, we highlight the features of some approaches, and present
two DSLs which are specific for the task of implementing type systems and
validation rules based on types.  While we concentrate on a language that we use
as a case study, all the issues we deal with throughout the paper are typical of
the task of implementing the type system of a programming language, thus the
presented techniques can be reused in other DSL implementations.

As a case study, we will use language for modeling entities and GUI forms to
edit them (Section~\ref{sec:casestudy}); although this is a toy language, still
it has many recurrent and interesting features to be dealt with in a type
system, like type inference, subtyping, entity inheritance, and requires some
validation rules relying on types.  In particular we first present the
implementation of the type system for this language in plain Java (by also using
Xtend2, a Java-like language shipped with Xtext) (Section~\ref{sec:plain-xtext})
and using Xbase (a reusable expression language, integrated with Java, shipped
with Xtext) (Section~\ref{sec:xbase}).
Then, we implement the type system for our case study DSL in Xsemantics
(Section~\ref{sec:xsemantics}) and in XTS (Xtext Type System)
(Section~\ref{sec:xts}) two DSLs for implementing type systems for Xtext.
We then evaluate all these mechanisms and hint contexts where one approach
might be better (and easier to use) than the others
(Section~\ref{sec:evaluation}).
While this paper focuses on static type checking, although the type systems
presented here may be used beyond that, e.g. for reduction rules and
interpreters.  The code of the implementations presented in this paper is
available at \url{https://github.com/markusvoelter/typesystemcomparison}.


