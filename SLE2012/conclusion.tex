\section{Conclusion}
\label{sec:conclusion}

In this paper we presented some approaches to develop type systems for DSLs
implemented in Xtext.  While a type system can always be implemented in plain
Java (possibly by relying on the extended features of Xtend), still it is
crucial for productivity to have a dedicated framework with specific
functionalities for implementing type systems.  Xtext provides the expression
language Xbase that can be reused in a DSL (together with its type system,
which provides a full integration with Java type system), but this will force
the DSL to be tightly coupled with Java, which is not always the desired choice.

Xsemantics and XTS both provide a framework and a DSL to make the implementation
of type systems much easier.  They allow the programmer to specify type system
rules in a declarative way, concentrating on the type computation and subtyping
relations aspects while hiding other details which do not target type systems.
The implementation of type systems will then be easier to maintain, since it
will not be spread through Java code.  From the type system specifications,
these tools generate the corresponding Java code both for the type system and
for the Xtext validator, thus the programmer will not have to duplicate efforts;
this will also enhance the consistency between type computation and errors due
to typing failures.  Indeed, both Xsemantics and XTS provide the functionalities
(presented in the Introduction) that we believe a type system framework should
provide.
