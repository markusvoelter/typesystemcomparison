\section{Comparison}

\begin{frame}[fragile,allowframebreaks]
  \frametitle{Which approach to use when?}
  
  Criteria to help in deciding which approach might be good for a particular use
  case:
  \begin{itemize}
    \item Context, e.g. coupling with Java
    \item Expressivity
    \item Verbosity/conciseness
    \item Customizability
    \item Additional features
    \item Target audience
    \item Learning curve
    \item Documentation
    \item Support
  \end{itemize}

\framebreak  
\begin{tabularx}{\linewidth}{ X   X }
\multicolumn{2}{c}{Coupling of the DSL with Java} \\ \hline
Java Type System & Other Type Systems \\ \hline
Xbase & Plain  \\
& Xsem  \\
& XTS  \\
\end{tabularx}

%\begin{itemize}
%  \item DSL may refer to \emph{JavaVMTypes} as known from Xtext 1
%   \item \emph{JavaVMTypes}: use \emph{import
%   ``http://\-www.eclipse.org/\-xtext/\-common/\-JavaVMTypes'' \ldots}
%  \item All approaches allow custom Java methods for type computation
% \item Xbase \emph{JvmModelInferrer} generates Java classes from DSL model
% elements
%\end{itemize}

\framebreak
\begin{tabularx}{\linewidth}{ l   X }
\multicolumn{2}{c}{Expressivity} \\ \hline
Plain & Java \\
Xbase & Xbase library \\
Xsem & similar to Xbase + specific syntax\\
XTS & EMF feature access + specific syntax\\
\end{tabularx}

% \framebreak
% \begin{tabularx}{\linewidth}{ l   X }
% \multicolumn{2}{c}{Verbosity/conciseness} \\ \hline
% Plain & $\approx$11KB \\% - \emph{TypeProvider}, \emph{TypeConformance},
% % validator \\
% Xbase &$\approx$5KB \\%- \emph{JvmModelInferrer}, validator\\
% Xsem & $\approx$5KB \\%- \emph{*.xsemantics} file, validator\\
% XTS & $\approx$4KB \\%- \emph{*.ts} file, validator\\
% \end{tabularx}
% 
% \begin{itemize}
%   \item Size is a potentially misleading metric! 
% \end{itemize}
% 
% 
% \framebreak
% \begin{tabularx}{\linewidth}{ l   X }
% \multicolumn{2}{c}{Customizability} \\ \hline
% Plain & - \\
% Xbase & Xbase framework \\
% Xsem & custom Java methods (generation gap pattern)\\
% XTS & custom Java methods (generation gap pattern) \\
% \end{tabularx}
% 
% \framebreak
% \begin{tabularx}{\linewidth}{ l   X }
% \multicolumn{2}{c}{Additional Features} \\ \hline
% Plain & - \\
% Xbase & generated Java classes for free \\
% Xsem & generic rules (not only typing), generates validators \\
% XTS & generates validators, recurrent typing patterns \\
% \end{tabularx}
% 
% \framebreak
% \begin{tabularx}{\linewidth}{ l   X }
% \multicolumn{2}{c}{Target audience} \\ \hline
% Plain & - \\
% Xbase & - \\
% Xsem & people familiar with formal systems \\
% XTS & - \\
% \end{tabularx}
% 
% \framebreak
% \begin{tabularx}{\linewidth}{ l   X }
% \multicolumn{2}{c}{Learning curve} \\ \hline
% Plain & Xtend \\
% Xbase & Xbase, \emph{JvmModelInferrer} methods \\
% Xsem & similar to rules written in commonly used type theory \\
% XTS & XTS documentation \\
% \end{tabularx}

% \framebreak
% \begin{tabularx}{\linewidth}{ l   X }
% \multicolumn{2}{c}{Documentation} \\ \hline
% Plain & Xtext/Xtend in general \\
% Xbase & Xtext documentation \emph{JvmModelInferrer} \\
% Xsem & http://xsemantics.sourceforge.net/ \\
% XTS & http://code.google.com/a/eclipselabs.org/p/xtext-typesystem/ \\
% \end{tabularx}
% 
% \framebreak
% \begin{tabularx}{\linewidth}{ l   X }
% \multicolumn{2}{c}{Support} \\ \hline
% Plain & - \\
% Xbase & - \\
% Xsem & - \\
% XTS & - \\
% \end{tabularx}

\note{
TODO
functional scalability: plain only has required features, not more
may scale to function rich implementation, xbase was developed with this
architecture, type inference and typesystem (generics) not possible with XTS

footprint: learn language, additional editors and dependencies
}

\end{frame}
