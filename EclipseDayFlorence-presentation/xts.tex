\section[XTS]{Implementation in XTS}

\begin{frame}
  \frametitle{Xtext Typesystem (XTS)}
  \tableofcontents[currentsection]
\end{frame}

\begin{frame}
  \frametitle{Xtext Typesystem (XTS)}
  \begin{itemize}
  \item Typesystem DSL written in Xtext
  \item Additional uses: interpreter, testing
  \item Recursive type computation
\end{itemize}
\end{frame}

\begin{frame}[fragile]
  \frametitle{The Plus Operator with XTS}

  \begin{itemize}
    \item Commonly used language elements 
  \end{itemize}

  \begin{footnotesize}
    % Generator: GNU source-highlight, by Lorenzo Bettini, http://www.gnu.org/software/src-highlite
\begin{tabular}[t]{l}
\noindent
\mbox{}\textbf{\textcolor{Plum}{subtype}}\ FloatType\ \textbf{\textcolor{Plum}{base}}\ IntType \\
\mbox{} \\
\mbox{}\textbf{\textcolor{Plum}{characteristic}}\ NUMERIC\ \{ \\
\mbox{}\ \ \ \ IntType,\ FloatType \\
\mbox{}\}\  \\
\mbox{} \\
\mbox{}\textbf{\textcolor{Plum}{typeof}}\ Plus\ -$>$\ \textbf{\textcolor{Plum}{common}}\ left\ right\ \{ \\
\mbox{}\ \ \ \ \textbf{\textcolor{Plum}{ensureType}}\ left\ :$<$=:\ StringType,\ \textbf{\textcolor{Plum}{char}}(NUMERIC) \\
\mbox{}\ \ \ \ \textbf{\textcolor{Plum}{ensureType}}\ right\ :$<$=:\ StringType,\ \textbf{\textcolor{Plum}{char}}(NUMERIC) \\
\mbox{}\ \ \ \ \textbf{\textcolor{Plum}{ensureCompatibility}}\ left\ :$<$=$>$:\ right \\
\mbox{}\} \\
\mbox{}
\end{tabular}

  \end{footnotesize}

  \note{
  \begin{itemize}
    \item subtype relationships
    \item grouping of types
    \item type computation, \emph{ensure}\ldots clauses 
  \end{itemize}}
  
\note{
\begin{frame}[fragile]
  \frametitle{The Plus Operator with XTS}
  \begin{itemize}
    \item Coercion of a number to a String
  \end{itemize}

  \begin{footnotesize}
    % Generator: GNU source-highlight, by Lorenzo Bettini, http://www.gnu.org/software/src-highlite
\begin{tabular}[t]{l}
\noindent
\mbox{}\textbf{\textcolor{Plum}{public}}\ EObject\ \textcolor{Black}{typeCoerce}(\ EObject\ candidateElement,\ FloatType\ candidate,\  \\
\mbox{}\ \ \ \ \ \ \ \ StringType\ expected,\ TypeCalculationTrace\ trace\ )\ \{ \\
\mbox{}\ \ \ \ \textbf{\textcolor{Plum}{if}}\ (\ candidateElement\ \textbf{\textcolor{Plum}{instanceof}}\ NumberLiteral\ )\ \{ \\
\mbox{}\ \ \ \ \ \ \ \ \ \ \ \ trace.\textcolor{Black}{add}(\ candidateElement,\ \textcolor{RoyalBlue}{"{}Number\ coerced\ to\ string."{}}); \\
\mbox{}\ \ \ \ \ \ \ \ \ \ \ \ \textbf{\textcolor{Plum}{return}}\ \textcolor{Black}{create}(cl.\textcolor{Black}{getStringType}()); \\
\mbox{}\ \ \ \ \} \\
\mbox{}\ \ \ \ \textbf{\textcolor{Plum}{return}}\ \textbf{\textcolor{Plum}{null}}; \\
\mbox{}\}
\end{tabular}

  \end{footnotesize}

  \begin{itemize}
    \item XTS will try custom coercion methods before throwing a type error
    \item coercion methods use polymorphic dispatch 
  \end{itemize}
  public EObject typeCoerce( EObject candidateElement, FloatType candidate, 
		StringType expected, TypeCalculationTrace trace ) {
	if ( candidateElement instanceof NumberLiteral ) {
			trace.add( candidateElement, "Number coerced to string.");
			return create(lang.getStringType());
	}
	return null;
}
\end{frame}}	
\end{frame}