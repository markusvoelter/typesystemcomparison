\section[XTS]{Implementation in XTS}

\begin{frame}
  \frametitle{Xtext Typesystem (XTS)}
  \tableofcontents[currentsection]
\end{frame}

\begin{frame}
  \frametitle{Xtext Typesystem (XTS)}
  \begin{itemize}
  \item A Java framework to declaratively specify type system rules 
  \item DSL to generate the Java code
   \item If no suitable declarative abstraction is available, 
   Java code can be added manually.
   \item \texttt{ITypesystem} interface with methods for
   \begin{itemize}
	\item calculating the type of model elements
	\item for comparing types for
	compatibility and subtyping relationships
   \end{itemize}
\end{itemize}
\end{frame}

\begin{frame}[fragile]
  \frametitle{Setting up}


  \begin{footnotesize}
    % Generator: GNU source-highlight, by Lorenzo Bettini, http://www.gnu.org/software/src-highlite
\begin{tabular}[t]{l}
\noindent
\mbox{}\textbf{\textcolor{Plum}{typesystem}}\ org.typesys.xts.guidsl.typesys.GuiDlsTypesystem\  \\
\mbox{}\ \ \ \ \textbf{\textcolor{Plum}{ecore}}\ \textbf{\textcolor{Plum}{file}}\ \textcolor{RoyalBlue}{"{}platform:/.../GuiDsl.ecore"{}} \\
\mbox{}\ \ \ \ \textbf{\textcolor{Plum}{language}}\ \textbf{\textcolor{Plum}{package}}\ org.typesys.xts.guidsl.guiDsl.GuiDslPackage\  \\
\mbox{}
\end{tabular}

  \end{footnotesize}


the platform URI for the Ecore file (the metaclasses for
which we want to specify the type system rules)

\end{frame}

\begin{frame}[fragile]
  \frametitle{Specifying types}

  \begin{footnotesize}
    % Generator: GNU source-highlight, by Lorenzo Bettini, http://www.gnu.org/software/src-highlite
\begin{tabular}[t]{l}
\noindent
\mbox{}\textbf{\textcolor{Plum}{section}}\ \textcolor{RoyalBlue}{"{}Types"{}} \\
\mbox{}\ \ \ \ \textbf{\textcolor{Plum}{typeof}}\ Type+\ -$>$\ \textbf{\textcolor{Plum}{clone}} \\
\mbox{}\ \ \ \ \textbf{\textcolor{Plum}{subtype}}\ IntType\ \textbf{\textcolor{Plum}{base}}\ FloatType
\end{tabular}

  \end{footnotesize}

\onslide<2->

  \begin{footnotesize}
    % Generator: GNU source-highlight, by Lorenzo Bettini, http://www.gnu.org/software/src-highlite
\begin{tabular}[t]{l}
\noindent
\mbox{}\textbf{\textcolor{Plum}{section}}\ \textcolor{RoyalBlue}{"{}Literals"{}} \\
\mbox{}\ \ \ \ \ \ \textbf{\textcolor{Plum}{typeof}}\ StringLiteral\ -$>$\ StringType \\
\mbox{}\ \ \ \ \ \ \textbf{\textcolor{Plum}{typeof}}\ BooleanLiteral\ -$>$\ BooleanType \\
\mbox{}\ \ \ \ \ \ \textbf{\textcolor{Plum}{typeof}}\ NumberLiteral\ -$>$\ \textbf{\textcolor{Plum}{javacode}}
\end{tabular}

  \end{footnotesize}

\onslide<3->

\medskip

	Recursive computation

  \begin{footnotesize}
    % Generator: GNU source-highlight, by Lorenzo Bettini, http://www.gnu.org/software/src-highlite
\begin{tabular}[t]{l}
\noindent
\mbox{}\textbf{\textcolor{Plum}{typeof}}\ AttributeRef\ -$>$\ \textbf{\textcolor{Plum}{feature}}\ attr
\end{tabular}

  \end{footnotesize}


\end{frame}

\begin{frame}[fragile]
  \frametitle{Specifying types}

  \begin{footnotesize}
    % Generator: GNU source-highlight, by Lorenzo Bettini, http://www.gnu.org/software/src-highlite
\begin{tabular}[t]{l}
\noindent
\mbox{}\textbf{\textcolor{Plum}{characteristic}}\ COMPARABLE\ \{ \\
\mbox{}\ \ \ \ IntType,\ FloatType,\ BooleanType,\ StringType \\
\mbox{}\}\ \  \\
\mbox{}\textbf{\textcolor{Plum}{characteristic}}\ NUMERIC\ \{ \\
\mbox{}\ \ \ \ IntType,\ FloatType \\
\mbox{}\}\ 
\end{tabular}

  \end{footnotesize}

\onslide<2->

  \begin{footnotesize}
    % Generator: GNU source-highlight, by Lorenzo Bettini, http://www.gnu.org/software/src-highlite
\begin{tabular}[t]{l}
\noindent
\mbox{}\textbf{\textcolor{Plum}{typeof}}\ Comparison\ -$>$\ BooleanType\ \{ \\
\mbox{}\ \ \ \ \ \textbf{\textcolor{Plum}{ensureType}}\ left\ :$<$=:\ \textbf{\textcolor{Plum}{char}}(COMPARABLE) \\
\mbox{}\ \ \ \ \ \textbf{\textcolor{Plum}{ensureType}}\ right\ :$<$=:\ \textbf{\textcolor{Plum}{char}}(COMPARABLE) \\
\mbox{}\ \ \ \ \ \textbf{\textcolor{Plum}{ensureCompatibility}}\ left\ :$<$=$>$:\ right \\
\mbox{}\} \\
\mbox{}\textbf{\textcolor{Plum}{typeof}}\ Minus\ -$>$\ \textbf{\textcolor{Plum}{common}}\ left\ right\ \{ \\
\mbox{}\ \ \ \ \textbf{\textcolor{Plum}{ensureType}}\ left\ :$<$=:\ \textbf{\textcolor{Plum}{char}}(NUMERIC) \\
\mbox{}\ \ \ \ \textbf{\textcolor{Plum}{ensureType}}\ right\ :$<$=:\ \textbf{\textcolor{Plum}{char}}(NUMERIC) \\
\mbox{}\ \ \ \ \textbf{\textcolor{Plum}{ensureCompatibility}}\ left\ :$<$=$>$:\ right \\
\mbox{}\}
\end{tabular}

  \end{footnotesize}

\end{frame}

\begin{frame}[fragile]
  \frametitle{Specifying types}

  \begin{footnotesize}
    % Generator: GNU source-highlight, by Lorenzo Bettini, http://www.gnu.org/software/src-highlite
\begin{tabular}[t]{l}
\noindent
\mbox{}\textcolor{Green}{//\ text\ widgets\ may\ only\ refer\ to\ non-boolean\ attributes\ } \\
\mbox{}\textbf{\textcolor{Plum}{typeof}}\ TextWidget\ -$>$\ \textbf{\textcolor{Plum}{feature}}\ attr\ \{ \\
\mbox{}\ \ \ \ \textbf{\textcolor{Plum}{ensureType}}\ length\ :$<$=:\ IntType \\
\mbox{}\ \ \ \ \textbf{\textcolor{Plum}{ensureType}}\ attr\ :$<$=:\ StringType,\ IntType,\ FloatType \\
\mbox{}\ \ \ \ \textbf{\textcolor{Plum}{ensureType}}\ validate\ :$<$=:\ BooleanType \\
\mbox{}\}\ \  \\
\mbox{} \\
\mbox{}\textcolor{Green}{//\ checkbox\ widgets\ may\ only\ refer\ to\ boolean\ attributes} \\
\mbox{}\textbf{\textcolor{Plum}{typeof}}\ CheckBoxWidget\ -$>$\ \textbf{\textcolor{Plum}{feature}}\ attr\ \{ \\
\mbox{}\ \ \ \ \textbf{\textcolor{Plum}{ensureType}}\ attr\ :$<$=:\ BooleanType \\
\mbox{}\ \ \ \ \textbf{\textcolor{Plum}{ensureType}}\ validate\ :$<$=:\ BooleanType \\
\mbox{}\}
\end{tabular}

  \end{footnotesize}

\end{frame}

% \begin{frame}[fragile]
%   \frametitle{The Plus Operator with XTS}
% 
% 
%   \begin{footnotesize}
%     % Generator: GNU source-highlight, by Lorenzo Bettini, http://www.gnu.org/software/src-highlite
\begin{tabular}[t]{l}
\noindent
\mbox{}\textbf{\textcolor{Plum}{subtype}}\ FloatType\ \textbf{\textcolor{Plum}{base}}\ IntType \\
\mbox{} \\
\mbox{}\textbf{\textcolor{Plum}{characteristic}}\ NUMERIC\ \{ \\
\mbox{}\ \ \ \ IntType,\ FloatType \\
\mbox{}\}\  \\
\mbox{} \\
\mbox{}\textbf{\textcolor{Plum}{typeof}}\ Plus\ -$>$\ \textbf{\textcolor{Plum}{common}}\ left\ right\ \{ \\
\mbox{}\ \ \ \ \textbf{\textcolor{Plum}{ensureType}}\ left\ :$<$=:\ StringType,\ \textbf{\textcolor{Plum}{char}}(NUMERIC) \\
\mbox{}\ \ \ \ \textbf{\textcolor{Plum}{ensureType}}\ right\ :$<$=:\ StringType,\ \textbf{\textcolor{Plum}{char}}(NUMERIC) \\
\mbox{}\ \ \ \ \textbf{\textcolor{Plum}{ensureCompatibility}}\ left\ :$<$=$>$:\ right \\
\mbox{}\} \\
\mbox{}
\end{tabular}

%   \end{footnotesize}
% 
%   \note{
%   \begin{itemize}
%     \item Commonly used language elements 
%   \end{itemize}
%   \begin{itemize}
%     \item subtype relationships
%     \item grouping of types
%     \item type computation, \emph{ensure}\ldots clauses 
%   \end{itemize}}
% \end{frame}
% 
% \note{
% \begin{frame}[fragile]
%   \frametitle{The Plus Operator with XTS}
%   \begin{itemize}
%     \item Coercion of a number to a String
%   \end{itemize}
% 
%   \begin{footnotesize}
%     % Generator: GNU source-highlight, by Lorenzo Bettini, http://www.gnu.org/software/src-highlite
\begin{tabular}[t]{l}
\noindent
\mbox{}\textbf{\textcolor{Plum}{public}}\ EObject\ \textcolor{Black}{typeCoerce}(\ EObject\ candidateElement,\ FloatType\ candidate,\  \\
\mbox{}\ \ \ \ \ \ \ \ StringType\ expected,\ TypeCalculationTrace\ trace\ )\ \{ \\
\mbox{}\ \ \ \ \textbf{\textcolor{Plum}{if}}\ (\ candidateElement\ \textbf{\textcolor{Plum}{instanceof}}\ NumberLiteral\ )\ \{ \\
\mbox{}\ \ \ \ \ \ \ \ \ \ \ \ trace.\textcolor{Black}{add}(\ candidateElement,\ \textcolor{RoyalBlue}{"{}Number\ coerced\ to\ string."{}}); \\
\mbox{}\ \ \ \ \ \ \ \ \ \ \ \ \textbf{\textcolor{Plum}{return}}\ \textcolor{Black}{create}(cl.\textcolor{Black}{getStringType}()); \\
\mbox{}\ \ \ \ \} \\
\mbox{}\ \ \ \ \textbf{\textcolor{Plum}{return}}\ \textbf{\textcolor{Plum}{null}}; \\
\mbox{}\}
\end{tabular}

%   \end{footnotesize}
% 
%   \begin{itemize}
%     \item XTS will try custom coercion methods before throwing a type error
%     \item coercion methods use polymorphic dispatch 
%   \end{itemize}
%   public EObject typeCoerce( EObject candidateElement, FloatType candidate, 
% 		StringType expected, TypeCalculationTrace trace ) {
% 	if ( candidateElement instanceof NumberLiteral ) {
% 			trace.add( candidateElement, "Number coerced to string.");
% 			return create(lang.getStringType());
% 	}
% 	return null;
% }
% \end{frame}}